\section{Results}

\subsection{Training comparison}
\begin{figure}[H]
	\centering
	\includegraphics [scale = 0.5]{Images/RL_GA_comparison_avg.png}
	\caption{placeholer}
	\label{figAVG}
\end{figure}

\begin{figure}[H]
	\centering
	\includegraphics [scale = 0.5]{Images/RL_GA_comparison_max.png}
	\caption{placeholer}
	\label{figMAX}
\end{figure}



\subsection{final model comparison}

\begin{figure}[H]
	\centering
	\includegraphics [scale = 0.7]{Images/diff.png}
	\caption{Difference between the two state action tables obtained using GA and RL. The black pixels represents the states where choosen action is the same, white pixels represent a different action's choice. The x-axis contains all the possible pairs of the first two elements of the state's 4D-vector, $(s_1,s_2)$, y-axis all the possible pairs of the last two elements $(s_3,s_4)$. }
	\label{figTABLEDIFF}
\end{figure}


\begin{figure}[H]
	\centering
	\includegraphics [width=0.5\textwidth]{Images/GAvRL_backup.png}
	\caption{The plot shows the results obtained testing the two models on the same test set. The x-axis represents the number of the test set, the y-axis the number of steps the model was able to take before reaching the goal. The blue line represents the results obtained using the model trained with the GA, the red line the results obtained using the model trained with the RL.}
	\label{figRLvsGA}
\end{figure}
% % Delete the text and write your Results here:
% %------------------------------------

% The results section can be combined with the discussion if appropriate. In case of many sub\-/experiments % \-/ fixes the problem that LaTeX won't hyphenate words with dashes in them.
% where the results are vaguely related or unrelated, it would be appropriate to combine the results and discussion. This way you have the information related to each sub-experiment gathered in one place. \par
% Provide uncertainties for the results, but don't discuss it. Do not involve personal opinions, just present the cold hard results in form of numbers, tables, graphs and some sentences. \par
% \Cref{tab:Some-numbers} shows a nice table with comma alignment. \par

% \begin{table}[htb]%
% \centering
% \caption{Table with comma alignment.}
% 	\label{tab:Some-numbers}
% 	\begin{tabular}{SSS} 		% S = special column format from the siunitx package. Aligns commas.
% 		\toprule
% 		{$m$}  &  {$a$}  & {$F$}  \\
% 		{(\si{kg})} &  {(\si{m.s^{-2}})} & {(\si{N})}  \\
% 		\midrule
% 		1,2 & 10,1 & 12 \\
% 		2,44 & 6,92 & 16,88 \\
% 		10 & 1,0 & 10 \\
% 		8,2 & 1,1 & 9,0 \\
% 		100 & 1 & 100 \\
% 		\bottomrule
% 	\end{tabular}
% \end{table}


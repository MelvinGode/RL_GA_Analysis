\begin{frontmatter}
%
% Title:
%------------------------------------
\title{%
Finding a balance between\\reinforcement and evolution\\
%\small For Physics Reports  % A good idea is to have the subject code and name as subtitle
}
%
% Authors:
%------------------------------------
% List an author with name ' Firstname Middlename Lastname ' like this:
% F. M. Lastname
\author[UppsalaUniversity]{Filippo Balzarini} 
\author[UppsalaUniversity]{Jason Kaxiras}
\author[UppsalaUniversity]{Melvin Gode}
\address[UppsalaUniversity]{Department of Computer Science, Uppsala University, Uppsala, Sweden}
%
% Date:
%------------------------------------
%
\newdate{dateName}{23}{05}{2024} % edit the date here, ' dateName ' has to match on these two lines.
\renewcommand*{\today}{\MonthYearDateFormat\displaydate{dateName}} 
% Options for displaying date: \MonthYearDateFormat,  \DayMonthYearDateFormat or \YearDateFormat
%
% Abstract:
%------------------------------------
\NameOfAbstract{Abstract} % Change abstract title here. If you write in Norwegian, write 'Sammendrag' (nb) or 'Samandrag' (nn)
\begin{abstract}
Machine learning includes a variety of techniques for solving specific problems. This paper explores the comparative advantages and disadvantages of using a genetic algorithm versus a reinforcement learning approach for the pole balancing problem. Both methods were subjected to identical training and testing conditions to ensure a fair comparison. The results demonstrated that reinforcement learning, due to its simplicity of implementation and superior environmental comprehension, outperformed the genetic algorithm in this scenario. This finding reinforces the prevalent use of reinforcement learning in similar contexts and highlights its potential for future applications in dynamic environments.

\end{abstract}
%
\end{frontmatter}
%
%
% Table of contents:
%------------------------------------
% If the report is very long for some reason (over 4 or 5 pages), use a table of contents.
% Uncomment everything below the line ---- to get table of contents (ctrl + /) (the / on numberpad):
%-------------
%
% \ 
% \vspace{1cm}

% \begin{minipage}{\textwidth}
%     \tableofcontents
% \end{minipage}
% \clearpage
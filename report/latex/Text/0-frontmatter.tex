\begin{frontmatter}
%
% Title:
%------------------------------------
\title{%
Comparative Analysis \\ of Genetic Algorithms
and Reinforcement Learning\\
%\small For Physics Reports  % A good idea is to have the subject code and name as subtitle
}
%
% Authors:
%------------------------------------
% List an author with name ' Firstname Middlename Lastname ' like this:
% F. M. Lastname
\author[UppsalaUniversity]{Filippo Balzarini} 
\author[UppsalaUniversity]{Jason Kaxiras}
\author[UppsalaUniversity]{Melvin Gode}
\address[UppsalaUniversity]{Department of Computer Science, Uppsala University, Uppsala, Sweden}
%
% Date:
%------------------------------------
%
\newdate{dateName}{23}{05}{2024} % edit the date here, ' dateName ' has to match on these two lines.
\renewcommand*{\today}{\MonthYearDateFormat\displaydate{dateName}} 
% Options for displaying date: \MonthYearDateFormat,  \DayMonthYearDateFormat or \YearDateFormat
%
% Abstract:
%------------------------------------
\NameOfAbstract{Abstract} % Change abstract title here. If you write in Norwegian, write 'Sammendrag' (nb) or 'Samandrag' (nn)
\begin{abstract}
% Delete the text and write your abstract here:
%------------------------------------

The abstract should be \textit{very} brief, two or three sentences may be enough. It must answer the following questions, however:
1. What did you do (What did you measure)?
2. How did you do it (which method)?
3. What did you discover (what was the results of the experiment)?
Results in form of numbers should be accompanied by an error: $R=\SI{3,05(2)e-6}{kg/s^{2}}$

\end{abstract}
%
\end{frontmatter}
%
%
% Table of contents:
%------------------------------------
% If the report is very long for some reason (over 4 or 5 pages), use a table of contents.
% Uncomment everything below the line ---- to get table of contents (ctrl + /) (the / on numberpad):
%-------------
%
% \ 
% \vspace{1cm}

% \begin{minipage}{\textwidth}
%     \tableofcontents
% \end{minipage}
% \clearpage